% find this class at https://archive.danleonard.us/scholarship/coursework/coursework.cls
\documentclass[basque,american]{../../../coursework}

\title{The Familiar Path of ETA}
\subtitle{}
\shorttitle{}
\author{Daniel}{Glenn}{Leonard}
\newdate{date}{20}{11}{2017}
\date{\displaydate{date}}
\course{PS}{300}{The Character and History of Radical Terrorism since the Mid-Nineteenth Century}{University of Illinois at Urbana-Champaign}
\instructor{Dr}{John}{A.}{}{Lynn}


\keywords{}
\addbibresource[location=remote]{https://archive.danleonard.us/scholarship/coursework/illinois/PS/300/eta.bib}
\baseurl{https://archive.danleonard.us/scholarship/coursework/illinois/PS/300/eta.xhtml}

\begin{document}
\setcounter{page}{6}
\maketitle

\foreignlanguage{basque}{Euskadi Ta Askatasuna}, or ETA, was a Basque
separatist group that engaged in armed insurrection against the Spanish state
until its disarmament in 2017. In its armed lifetime, ETA militants killed
hundreds of Spanish security forces and civilians alike. ETA's persistence and
ferocity, not seen elsewhere within Spain, has echoes of and often directly
mimicked the strategies and philosophies within the various forms of the Irish
Republican Army. ETA and the IRA saw both formed in response to imperialism,
represented a disaffected agrarian society, employed cellular guerilla
warfare, and had strong left-wing goals for a future independent state.

While ETA was formed in 1959, the roots of Basque nationalism run much deeper.
The Basque people, while residents of France and Spain, are entirely unrelated
to Indo-European peoples linguistically and genetically. A 2015 study was able
to identify an independent lineage tracing back at least to Neolithic farmers
and hunter-gatherers inhabiting the same region 5,500 years ago
\parencite{Guenther2015}. Basque nationalism was thus understood by its early
adherents within a racial context, distinct from other European independence
movements against an ethnically related state such as in Catalonia or Ireland.

While the Basque region had been part of empires and kingdoms since Roman
conquest circa 50~BCE, the rural social organization consisting of farmers and
fishermen allowed Basques to stay quiet and self-sufficient underneath the
kingdoms of the middle ages \parencite[31]{Watson2007}. The kingdoms of
Castile and Navarre codified the relationships with their respective Basque
regions under a system of fueros, charters allowing independent, autonomous
regional councils that could tax and mobilize soldiers in return for their
recognition of the respective Crown. This historic and ancient ideal of
autonomy, and especially the well-documented fueros, would be a driving force
for Basque nationalism.

The rise of economic liberalism in the eighteenth and nineteenth centuries was
tied to a growing merchant class across the Iberian Peninsula. Moneyed
liberals who sought to centralize their power over commerce opposed the system
of fueros that gave Basques autonomy over much of their trade. Carlism emerged
as a reactionary ideology against secularism and capitalism, and Basques
overwhelmingly fought during the Carlist Wars to maintain the previous
political system. Following liberal victory, the new Spanish state proceeded
to abolish the fueros in 1876 and centralized all administration in Madrid. 

In Restoration Spain following the last Carlist War, Basque nationalism was
thus a very reactionary and traditionalist movement. Sabio Arana emerged at
the end of the nineteenth century and remains the father of modern Basque
nationalism. He is credited with the creation of a flag and name for a
unified Basque region and formed a political party, the PNV, which remains a
major force in regional politics. Arana viewed independence along the lines of
Catholicism and Basque racial superiority over Indo-Europeans. Even with his
prolific writings, Arana had little success achieving any Basque autonomy in
his lifetime. However, the PNV and its use of racial identity was the root of
nationalism in the Basque region until the emergence of ETA.

The PNV fought with Republicans during the Spanish Civil War, conceding their
Catholicism to a primacy of regional autonomy and out of fear of Francisco
Franco. It was this war that altered the trajectory of Basque nationalism and
caused it to take a left-wing course for the remainder of its existence.
General Franco and the German Nazis saw the Basque region as a hotbed for
communism, and razed the historically significant yet strategically
unimportant Basque city of Guernica on its traditional market day as a
\enquote{test [of the] young Luftwaffe} \parencite{Goring1946}. While the
Basque militias were decimated and quickly lost to Franco's advance, the
thousands who perished in Guernica became martyrs for Basque nationalism and
Pablo Picasso's \emph{Guernica} remains an international symbol of the horrors
of war.

Francoist Spain saw centralization of economic prosperity in Madrid at the
expense of the rest of the country, especially rural areas. Fixed price
controls implemented early on cut agricultural wages 40\% over the first ten
years of the dictatorship, especially impactful in the primarily-rural Basque
regions \parencite[169]{Watson2007}. An economic insecurity not previously
seen gripped the Basque Country as agricultural workers swarmed to the cities
in search of work.

Simultaneously, General Franco prohibited the use of regional and foreign
languages, punishable by imprisonment and often death. The suppression of
language was so strong that many Basque parents opted to cease speaking the
language in their own homes, resulting in a generation without the ability to
speak the native language of their parents. Between 1930 and 1970, the
percentage of \foreignlanguage{basque}{Euskara} speakers in the Basque Country
dropped from 41.7\% to 19.8\% \parencite{Clark1981}. It was this generation of
Basques, who had lost their traditional language growing up in a climate of
oppression, who would begin the clandestine movements against Franco.

ETA was formed in 1959 by leftist students who had been active throughout the
1950s. In a stark turn from previous Basque movements that had been
reactionary and Catholic, ETA's ideology was staunchly Marxist in response to
General Franco's religious-nationalist dictatorship. Instead of Basque
identity originating from race, the students saw the Basque language as the
primary point of their identity, and its preservation as crucial to the
establishment of their autonomy \parencite[15]{Muroa2017}. ETA rejected the
PNV's goal of peaceful resistance, stating from the beginning their intent to
use armed insurrection where necessary. Their first attack occurred in 1961,
where ETA attempted to derail a train. They did not intend to nor actually
kill anyone, yet the Franco regime tortured over a hundred people in response.

The early years of ETA produced martyrs and significant popular support. The
organization's first killing was in 1968 of a police officer who had initiated
a traffic stop on ETA member Txabi Etxebarrieta. Etxebarrieta shot and killed
the officer, but was himself shot by police within hours
\parencite[17]{Muroa2017}. ETA almost immediately transitioned to planned
violence against the state, and became the major force of Basque separatism.

The next killing was a planned assassination: that of ruthlessly brutal police
commander Melitón Manzanas. The state responded with a significant crackdown.
That year, 1,953 Basques were detained \parencite[43]{Whitfield2014} and
factions soon arose within ETA. Precipitating future intra-organization
conflict, and reflecting many divisions within past groups, some in ETA wanted
to focus exclusively on military action and others felt it better to turn
towards worker's rights in an age of right-wing corporatism. The leftists
became known as ETA~(VI) while the nationalist minority became ETA~(V)
\parencite[18]{Muroa2017}. At the same time, the Irish Republican Army was
suffering its own divisions. The Provisional IRA and the Official IRA broke
apart, the former criticizing the latter's Marxist view of a working-class
fight against the state, instead seeking a return to the sectarian
Catholic/Protestant fight that had defined previous eras in Ireland. And just
as the militant Provos became the more well-known in Ireland, ETA~(V) quickly
became the primary faction as ETA~(IV) split further and dissolved into
Marxist political parties.

In 1970, ETA (V)'s leadership was arrested and sentenced to death, which
provoked Pope Paul~IV's condemnation and international sympathy, leading to
commutation of the death sentences. Within this international blowback towards
Franco, ETA was able to grow significantly in power. Three years later, ETA
managed to execute the assassination of Prime Minister Luis Carrero Blanco. As
Generalissimo Franco's health was in decline, Carrero was seen as an impending
successor to the dictator. The attack sent the government into crisis, and
solidified ETA's reputation within the Basque Country and elsewhere in Spain
as the strongest opposition to the regime; however, this led to much more
violent government repression \parencite[44]{Whitfield2014}. Franco expanded
capital punishment in response, and the final use of the death penalty in
Spain included two ETA members.

1974 saw yet another split in ETA, an intra-organizational conflict bubbling
up from years previously. The majority sought political participation to
accompany armed struggle as a means to independence, and was known as
ETA-pm~(\foreignlanguage{basque}{politico-militar}). Others favored continuing
a solely militaristic fight against the government as
ETA-m~(\foreignlanguage{basque}{militar}) and eschewing political processes,
again reflecting a long-standing division in both ETA and its Irish analogue.
Around the same time, Sinn Féin, the political equivalent of the IRA, debated
whether to end their parliamentary abstentionism and Official Sinn Féin split
off to engage in politics.

Francisco Franco died in his bed in 1975, and Spain soon began its transition
to democracy. ETA -- in both its factions -- was actually most violent during
this period, countering expectations that the end of Francoism would satiate
the separatists. In fact, the transition saw political centralization across
ethnic divides. The Spanish Socialist Workers' Party~(PSOE), which had in 1974
supported self-determination for ethnic regions, dropped all calls for
regional independence as it sought power in the new republic. The Basque
Socialist Party similarly dropped its calls for independence in 1978
\parencite[47]{Whitfield2014}. As the PSOE went on to form the government in
the early 1980s, it is readily seen how many Basque nationalists viewed these
moves as cynical attempts to gain political power at the expense of ideology.
Further hurting the Basque cause, the Spanish state did not -- and still
refuses to -- reflect on the crimes committed during the Francoist era, a
standard procedure in countries abolishing dictatorships. Just after the first
election, Spain passed the 1977 Amnesty Law, exonerating all crimes committed
during Franco's regime. The United Nations continues to argue the Law
constitutes a violation of international human rights law, as \enquote{the
main obstacle in the way of opening investigations and criminal proceedings
with respect to serious human rights and humanitarian law violations}
\parencite{Greiff2014}. For a people who were criminalized for speaking
their mother tongue, it is not difficult to imagine frustration at the
awarding of amnesty to their oppressors. In the preceding referendum proposed
to initiate the transition to democracy, all opposition political parties
called for abstention as the vote was seen as a ploy by the authoritarian
successor to Franco. While only a quarter of Spaniards abstained, almost half
of Basques refused to vote, regarding the referendum as simply a part of the
colonial Spanish state. This is distinctly similar to an early era of Irish
Republicanism, during the creation of the Irish Free State. Although it was
afforded a parliament with home rule, Republicans opposed to the Anglo-Irish
Treaty objected to the absence of Ulster and the status as a dominion of the
British Empire. Sinn Féin thus took an abstentionist policy, and refused
participation in both the Irish Dáil and the British Parliament.

From democratization onward, ETA adopted a \enquote{long war} strategy, and a
majority of its victims were claimed under democracy. It is this period that
is strikingly similar to the Provisional IRA, in that both used the same
strategy and types of attack -- notably car bombs. For ETA members, prison
became a regular facet of life, and banners calling for amnesty flew across
the Basque Country just as murals of H-Block prisoners did in Ireland. But ETA
was not simply mimicking the IRA; rather, the two organizations maintained
clandestine links and discussion over decades. While neither organization has
confirmed, it is suspected that the IRA was a significant source of ETA's
weaponry, especially following the former's disarmament.
\textcite{Whitfield2014} interviewed a former Basque nationalist Member of the
European Parliament, who described the Provisionals' war as a \enquote{brother
conflict} and, ironically, saw the Basque issue as easier to solve compared to
the sectarianism in the Irish Troubles.

The IRA's most visible and direct communication with ETA, however, came after
the Good Friday Agreement and the end of the Provisionals. ETA's political
wing and Sinn Féin equivalent, \foreignlanguage{basque}{Herri Batasuna},
maintained contact with its Irish counterpart for the duration of the peace
talks. Following the Good Friday Agreement, \foreignlanguage{basque}{Batasuna}
held a 1998 conference to discuss what of the Irish cessation of violence
could be applied to the Basques \parencite[45]{Muroa2017}. Even though
disarmament wouldn't come for two decades, the Good Friday Agreement was a
major turning point towards peace in the Basque Country. While ETA would
renege on ceasefires declared in 1998 and 2006, they also held many peace
talks with Spain mediated by the United Kingdom and prominent members of Sinn
Féin Gerry Adams and Gerry Kelly during this same period
\parencite[162]{Whitfield2014}. In 2010, peace was once again on the agenda
and Gerry Adams wrote in an editorial in the Guardian \enquote{this is an
important development that creates an opportunity for an end to conflict in
the Basque country and for real political progress \dots Sinn Féin will
promote conflict resolution and assist in whatever way we can}
\parencite{Adams2010}. In September of that year, ETA declared what was to be
their final ceasefire. After nearly 58 years, ETA fully disarmed on April
8\textsuperscript{th}, 2017.

It's clear that ETA and the IRA are uniquely bound by strategies, ideologies,
and histories. However, the cause of Basque nationalism ended with a different
result than Irish republicanism. With no treaty as in the Good Friday
Agreement, the disarmament of ETA came with little concession from the Spanish
state and many of ETA's main goals have not been achieved. For one, there
remains no independent Basque state, while the Republic of Ireland existed
throughout decades of Irish violence. The 1977 Amnesty Law still protects
Franco-era crimes against the Basque people, including torture and
extrajudicial killings, from being prosecuted. While the Basque language has
seen a resurgence, Franco's criminalization of its speakers remains in a
generation who speak it secondhand to their native-speaking children. The
Basque problem hardly solved and will likely remain for years to come, but it
would be worthwhile to look towards Ireland for advice.

\printbibliography

\end{document}