% find this class at https://archive.danleonard.us/scholarship/coursework/coursework.cls
\documentclass[american]{../../../coursework}

\title{East African Motherhood after Decolonization}
%\subtitle{}
\author{Daniel}{Glenn}{Leonard}
\DTMsavedate{date}{2019-12-13}
\date{\DTMUsedate{date}}
\course{ANTH}{438}{Primate Life History Evolution}{University of Illinois at Urbana-Champaign}
\instructor{Dr}{Kathryn}{B.}{H.}{Clancy}

\keywords{Kenya, Tanzania, internet, birth}
\addbibresource{motherhood.bib}
% \baseurl{https://archive.danleonard.us/scholarship/coursework/illinois/ANTH/438/motherhood.xhtml}

\begin{document}
% \setcounter{page}{60}
\maketitle

\section{Introduction}

In \citeauthor{Coo08}'s \citeyear{Coo08} report on syphilis in the
Baganda\footnote{This name follows Bantu ethnic group naming customs. An
individual is ``Muganda,'' the people ``Baganda,'' their language
``Luganda,'' and the country ``Buganda''} populace, he describes how this
``most naturally civilized, charming, kind, tactful, and courteous of black
people'' (p. 1022) is suffering from exceptionally high rates of syphilis,
averaging around half of the entire population. The cause of this epidemic,
so tragically inflicted on one of the continent's less-savage peoples, is
blamed squarely on the promiscuity women were free to practice in under the
liberation of British colonization. Missionaries had brought Christianity to
Buganda, freeing women from the tribal yoke of polygamy and subservience and
introducing the liberal Western ideals of turn-of-the-century sexuality.
Without the country's former legal punishments for unchastity, they became
``merely female animals with strong passions, to whom unrestricted
opportunities for gratifying these passions were suddenly afforded''
(p. 1023). The solution, he concluded, was in the imposition of Western
medicine, specifically outlined in stereotypically British-colonial terms as
via the use of local subservient chieftains to control the Baganda at large.

\textcite{Mus02} outlines how the colonial occupation in Buganda utilized the
mask of Western medicine to construct the ``other'' in Baganda women while
simultaneously using women as the harbinger for civilization and progress.
For the colonial medical establishment, the aforementioned promiscuity of
Baganda women was a cause of the population's decline, which was expected to
result in the nation's extinction. Doctors of the time used pelvic
measurements of Baganda women to demonstrate how local customs caused their
pelvises to become ``deformed'' in a way ill-suited for reproduction
\parencites[][]{Coo33}[in][100]{Mus02}. This crisis of underpopulation was
one only for the Western observer, to whom Buganda was the height of native
civilization -- being organized as a monarchy analogous to those of Europe --
and one whose extinction would be a loss to anthropology and to the greater
colonization efforts in British East Africa. In controlling women's actions
and creating otherized depictions of their bodies, the medical establishment
considered themselves to have ``sav[ed] a `race' that was in danger of going
extinct'' \parencite[109]{Mus02}.

\section{The Postcolonial Era}

The idea of certain civilizations living ``backwards'' lives in need of
improvement towards the Western ideal carried on through decolonization, and
in the middle of the twentieth century became the centerpiece of what would
come to be called modernization theory. Thinkers in this school placed the
impetus of development -- defined as the attaining of a Western industrial
state -- upon the peoples of the Third World. As \textcite{Tip73} identifies,
the language of the ``savage'' and the ``civilized man'' disappear, yet the
ideological foundation of nineteenth-century ethnocentrism remains.
Modernization theorists ``continue to evaluate the progress of nations \dots
by their proximity to the institutions of Western, and particularly
Anglo-American societies'' \parencite[206]{Tip73}. The new dichotomy is
``modern'' versus ``tradition,'' where formerly colonized societies must take
it upon themselves to eschew established customs and pull themselves up to
the rest of the world.

The modernization theory's implications extend far beyond the macroeconomic
structure of postcolonial states. Many scholars, dissenting from the
neoliberal consensus, have argued that the structural adjustment of the
previous half-century has been detrimental to the poorest inhabitants of the
developing world in less economically visible ways. \textcite{Ear96} argues
that declines in the provision of social services almost always result in an
increased load on women's labor. Using the example of water infrastructure in
decline in Tanzania, she argues that while dehydration impacts men and women
equally, it is women who make up for lack of water sources by traveling to
retrieve for their children potable water: the reallocation of social
resources ``contain[s] an implicit assumption that the process of social
reproduction which is carried out by women \textit{unpaid} will continue
regardless of the way the resources are allocated'' (p. 124, emphasis
original).

Discussion of the birth experience in sub-Saharan Africa rarely avoids
addressing health education in the region. Sex education in postcolonial 
Africa is frequently a top-down approach, mediated by the UN and national
governments, supplanting traditional forms of education. However, for many
African women, this messaging rarely reaches them for reasons of parental
inhibitions \parencite{Mbu07} or lack of funding \parencite{Agb11}. For a
third of women, initial knowledge about the pregnancy and birth process comes
from firsthand experience of witnessing birth rather than from friends,
doctors, or teachers \parencite{Cha90}. As literacy and internet access have
improved, so has access to sex education for many East Africans
\parencite{Mee95}. However, the focus of international and governmental
agencies toward funding programs such as provisioning family planning supplies
does not address the spread of misinformation about such supplies
\parencite{Chi10}. Notably, the most frequently-visited page on the Swahili
Wikipedia every month since January 2016 has been ``Ufahamu wa uwezo wa
kushika mimba'' (trans, ``fertility awareness'') \parencite{Wik}. The page has
since 2014 contained sections on the differing motility of X- and
Y-chromosome-containing sperm cells \parencite{Ufa141}, a belief for which no
evidence has been found \parencite{Hos01}. Efforts to encourage education in
developing nations must consider the diverse ways in which people in a very
heterogenous part of the world gain their information and adapt to those
challenges rather than top-down approaches.

Although education is certainly a laudable goal, the evidence-based scientific
establishment is not beyond reproach. Focuses on maternal health often exist
only as a proxy for concern over potential children. While some work has been
done to generate a holistic view of how community factors impact the health of
infants \parencite{Mag00}, \textcite{Rei19} describes how Western scientists'
focus on a child's growth serves to instrumentalize the mother's body. While
not as dehumanizing as the colonial-era pelvic measurements, the Gambian study
she describes serves a similar function in both removing African women's
agency and in maintaining Western conceptions of the female body.

In my own work on poverty amongst persons with disabilities in Kenya
\parencite{Leo19}, a refrain I found frequent was the intersection of
disability and single-motherhood. Nearly every mother I spoke to in a
disability community described having had their male partner flee upon news of
a pregnancy, leaving them to take care of the child. While considering the
state of single-motherhood as a ``problem'' in itself
\parencite[for an example, see][]{Oma95} ignores women's autonomy to choose
their family planning, the mothers I spoke to were unanimous in describing
their intersecting hardships of motherhood and disability. Women with
disabilities are rarely able to find independent employment, and social
services in Kenya are so lacking for both people with disabilities and single
mothers that most in the community I worked with never left abject poverty.

Another notable dynamic of postcolonial motherhood is the development of
long-distance alloparenting. The mechanization of agriculture and simultaneous
urban development brought by globalization have all but required many
Tanzanians to seek work not in their pastoralist villages but in the growing
cities \parencite{Coc19}. \textcite{Kyo15} describes how Tanzanian mothers --
unable to support their children in the expensive cities to which they
gravitate -- often leave children behind with grandparents or other kin, who
alloparent while the mother either gains a stable footing or remits money back
to the children and their caretakers. At the most extreme end, this distant
mothering becomes transnational, where African mothers move to Western nations
in a search for enough money for the allocare of their children. Such distant
separation, which \textcite{Kyo15} found could be up to seven years, was
emotionally difficult for these mothers, with one describing her coping
strategy as ``work[ing] so hard that I don't have idle time to think about how
I miss my children'' (p. 82).

While decolonization has brought with it an end to chattel-style control of
East African populations, mothers have still faced hardships in navigating a
rapidly changing world. The exploitation of women's bodies in the name of
science continues, albeit in a more humane form. International and national
efforts to manage population overlook women's actual needs and desires,
preferring the provisioning of condoms at public restrooms to real
culturally-sensitive education that takes persons with disabilities into
account. Western economic hegemony in the neocolonial era requires mothers to
rely on allocare just to gain access to the labor market. Even in the face of
these intersecting challenges, mothers in East Africa continue to persist and
make difficult decisions for themselves and for their children. An approach to
development and aid in this region must take mothers' perspectives into
account.

\printbibliography

\end{document}