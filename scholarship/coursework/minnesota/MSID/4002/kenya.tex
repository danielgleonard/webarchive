% find this class at https://archive.danleonard.us/scholarship/coursework/coursework.cls
\documentclass[american]{../../../coursework}

\title{Five Models for Development in Kenya}
\subtitle{}
\shorttitle{}
\author{Daniel}{Glenn}{Leonard}
\DTMsavedate{date}{2018-12-14} % all MSID papers submitted 2018-12-14T16:28:20+03:00
\date{\DTMUsedate{date}}
\course{MSID}{4002}{Country Analysis}{University of Minnesota Twin-Cities}
\instructor{Dr}{Fred}{Opiyo}{}{Jonyo}

\keywords{colonialism, Africa, policy}
\addbibresource{kenya.bib}
% \baseurl{https://archive.danleonard.us/scholarship/coursework/minnesota/MSID/4002/kenya.xhtml}

\begin{document}

\maketitle
% \setcounter{page}{23}
% \begin{abstract}
%     The legacy of colonization lives on in Kenya, where oppression runs
%     rampant by local elites as well as global capitalist institutions. Five
%     key areas where Kenya's social structure maintains colonialist social
%     relations are identified, and solutions are proposed for the detailed
%     problems.
% \end{abstract}

% \printkeywords

\section{Introduction}

The postcolonial situation of Kenya is marked with several crises. Despite
gaining independence over a half-century ago, the country remains in many ways
beholden to the colonial legacy. Tribalism is a major point of contention in
Kenyan politics, with members of its varying ethnic groups fighting for
supremacy \parencite{Orvis2001}. Land ownership is unequally concentrated in
the hands of a ruling bourgeois class, while many Kenyans lack access to land
\parencite{Syagga2006}. Further, the use of land is highly environmentally
unsustainable \parencite{Syagga2006}. Healthcare in Kenya is vastly
underdeveloped, with high death rates from preventable disease
\parencite{Feikin2011}. Finally, the relationship of Kenya to the West is
primarily extractive in nature, despite the attainment of independence.

\section{Kenyan Tribalism}
\subsection{Historical Context}

Kenya as a state traces its heritage to British East Africa, where arbitrary
borders were drawn across Africa by European powers in order to consolidate
their holdings. The Berlin Conference of the late nineteenth century
solidified the British Empire's borders as irrevocable and indisputable,
forming what would come to be called Kenya.

African political systems were not formed based on the Westphalian model of
European states, and as such were not equipped to deal with the imposition of
state violence and oppression. Ethnic groups were either cleaved in half
across international borders or were forced together regardless of historical
relations. Compounding this was the British system of divide and rule, in
which the colonial administration stoked resentments between the ethnic groups
within their territory in order to deflect violence from themselves
\parencite{Wesseling1996}.

After independence, tribal identity has remained a major part of Kenyan social
life. 42~languages are recognized within the relatively small nation, and all
share differing relations with one another. Post-colonial political elites in
Kenya have utilized this British-instilled ethnic divide for their own
political gain, using tribal nepotism to ensure their own supremacy
\parencite{Orvis2001}. Even archaeological research is influenced by the
elites' tribalism. Archaeology outside that of early human evolution has
failed to see the light of day, with most archaeologists being Europeans
following in the footsteps of the Louis Leakey and family \parencite{Koff1997}.
\textcite{Schmidt1995} has argued that this white-dominated focus on
paleoanthropology is an ideological choice by the government: the Kenyan
ruling elite is threatened by the land claims of the country's ethnic groups,
many of whom have long been evicted from their historical homelands.

\subsection{Solutions}

Tanzania, immediately south of Kenya, has had a vastly different story in
regards to tribalism, despite having similar societies and languages. In both
countries, the coast is occupied by the native Swahili population, who have
long been international traders and whose language (also known by its endonym
Kiswahili) has become the primary mode of state and interpersonal
communication throughout the region. While both states have adopted Swahili as
an official language, Kenya permits its ethnic groups to continue using the
language of their heritage, which has led to two-thirds of Kenyans having
fluency in a mother tongue alongside Swahili \parencite{Githiora2008}.
Conversely, the Tanzanian state has since independence implemented a
single-language policy \parencite{Topan2008}. This use of a single official
language has been cited as a major reason for Tanzania's relative lack of
intertribal relations. Tanzania's first president, Julius Nyerere, said as
much during his 1990 resignation from his position as party chairman,
declaring that ``making Kiswahili Tanzania's language has helped us greatly in
the battle against tribalism. If every Tanzanian had stuck to using his tribal
language … we would not have created the national unity we currently enjoy''
\parencites{Nyerere1990}[cited in][]{Topan2008}.

Nevertheless, Tanzania's effective abolition of its tribal languages comes
with significant consequences. The colonial system served as a weapon to
destroy cultural identity in the pursuit of capital. As such, it would behoove
one who desires a reversal of the colonial legacy to work toward preserving
cultural identity rather than hastening its death. Even where other tribal
cultures are taught in school, the language of instruction remains Swahili
\parencite{Topan2008}.

While it may be possible to reduce intertribal conflict via linguistic unity,
one must take care to preserve cultural heritage. If African societies are
crushed under unity, there would be no practical difference from the colonial
agenda.

\section{Land Ownership}
\subsection{Historical Context}

Western empires chose to colonize the world for a twofold purpose: land and
labor. Raw materials were rapidly diminishing in Europe, and its industry
needed fresh land for extraction. Thus, colonization sought to retrieve the
crops and materials needed by industrial capital to turn a continuous profit.
In doing so, the empires claimed only the highly-arable land, shunting the
indigenous population into those areas unsuitable for cash crops. In Kenya,
clandestine treaties with indigenous leaders gave Britain legitimacy over many
areas \parencite{Rutten1992}, and in others brute force was used. The Native
Lands Trust Board in 1930 declared that indigenous Kenyans may not hold land
in the way Europeans had the right to unilaterally claim territory for
plantations \parencite{Syagga2006}. Africans were sent into a state of
landlessness far away from their historical homelands.

Today, large swaths of African land are occupied by enormous plantations
occupying land to which many indigenous groups have historical claims. Kenya
has a sizable ruling class which profits from this land ownership. While many
Kenyans remain landless peasants, rich Kenyans were able to during
independence take over the colonial plantations, especially the political
dynasty of the Kenyatta family \parencite{Syagga2006}.

\subsection{Solutions}

In South Africa and Zimbabwe, grassroots movements have found footholds to
return African land to indigenous Africans. In these countries, the inequality
of land ownership during the colonial era was more severe than anywhere else
in Africa, and the relatively more difficult and longer-lasting struggle for
independence has meant that only in recent years have the countries begun to
reckon with European ownership of land \parencite{Syagga2006}.

This model used in southern Africa, of the state intervening in land
ownership, may be quite applicable to the situation of Kenya. While roundly
criticized by European settlers, the state can be recognized as the sole
arbiter of land rights -- ownership of real property is a social construct
that can be changed at will, not a concept beholden to laws of the Universe.

However, the Kenyan state has a conflict of interest in returning land rights.
While in South Africa and Zimbabwe the state represents the landless majority,
Kenyan political elites are largely profiteers of the present land ownership
system. Rather than an external and underrepresented minority owning most of
the country's farms, the current President of Kenya belongs to a family which
owns a large proportion of the country's large farms
\parencite{Berg-Schlosser1982}. It is unlikely that the Kenyan ruling class
would choose of their own volition to sacrifice their own bourgeois supremacy.
This was seen most clearly in the Maasai land movements of the twenty-first
century, where Maasai whose treaties with the British empire had expired chose
to petition the government for their land. Rather than follow the hundred-year
treaty, the Kenyan government chose to quash the Maasai rebellion.

Nevertheless, it may be possible to equalize land rights. Through grassroots
movements, the state may be forced to reckon with its control over land.
\textcite{Syagga2006} has proposed policies that the Kenyan state can be
forced to make, such as an amendment to the constitution allowing the seizure
of idle land. However, these proposals still do not address the desires of
ruling class, permitting them to be the implementers of such policy proposals.

\section{Environmental Unsustainability}
\subsection{Historical Context}

The colonial system utilized its possessions as extractive sources for
resources, especially in Kenya of tea and pyrethrum. The empire had no need to
maintain the ecological diversity of the country, as the consumers of colonial
exports had no connection to the land from which they were sourced. This
system has not declined in any form since independence, as land rights were
continued even in the independent state. Large plantations remain devoted to
single export crops, continuing the ecological degradation of the country
\parencite{Syagga2006}. As is well-known amongst ecologists, the reduction of
wild flora to make way for monocropping will inevitably lead to mass famine as
the natural ecosystem that supports farming is itself extinguished.

\subsection{Solutions}

Kenya must immediately work to not only preserve but enhance its natural
ecosystem. While the country currently has many game preserves and national
parks which preserve the ecosystem, they are few and far between. The colonial
system of enormous plantations will not survive inevitably, and Kenyans must
work toward a system that introduces the natural environment back into the
country.

However, it must be noted that this will be exceedingly difficult. Again, the
ruling class in Kenya is inexorably tied to continued profit from exclusive
colonial-style agriculture. It is not likely that they would give up their
rights to run their farms as they see fit, regardless of the ecological
consequences \parencite{Berg-Schlosser1982}. Secondly, Kenya is economically
dependent on its exports and cannot simply sacrifice those.

\section{Healthcare}
\subsection{Historical Context}

The British empire had little need to preserve Kenyans' physical health, as
they were solely interested in extraction. The little care that was given to
the indigenous was explicitly focused on ensuring their forced labor force
would be strong enough to continuously produce goods for export to the
imperial capital. For instance, state security would regularly conduct random
inspections on Kenyans' dwellings under the Public Health \& Prevention
Ordinance, with any evidence of mosquito habitation, exposed food, or simply
general uncleanliness invoking severe criminal punishments, often corporal
\parencite{Achola2010}. This policy was still unable to maintain wellness in
the African population, as the poverty imposed by the colonial administration
precluded subjects' ability to maintain their own health.

Today, Kenya's healthcare system has improved greatly since the colonial era.
The government operates a hierarchical system of hospitals throughout the
country, and most are able to access at least rudimentary care, often at low
cost. However, the system is very fractured, with a multi-tiered system of
government hospitals for the poor and private healthcare for the wealthy. Even
in government hospitals, services can be priced outside the budgets of many of
Kenya's poor even when such services are available. The country also suffers
from high rates of infectious disease, primarily targeting the poor who cannot
afford healthcare. For instance, children under 5 years of age in the
unregulated urban settlement of Kibera suffer from an average of 8 diarrhea
days per year \parencite{Feikin2011}.

\subsection{Solutions}

The single-payer healthcare model has worked extremely well in the Soviet
Union, Cuba, the United Kingdom, and Canada, where residents are free to
access the health system without worry of bankruptcy. In Kenya, this system is
absolutely necessary. As a tropical nation, infectious disease is an
omnipresent fact of life amongst Kenyans. Widespread poverty means that for
many health-related deaths, the root cause of death was a lack of money to
access the healthcare system.

The government, however, will find significant trouble in implementing such a
system. As Western nations manufacture the world's medical technology, the
threat of a large negotiator like a single-payer system usurping small
hospials and insurers threatens the profits of pharmaceutical and medical
technology corporations. Structural Adjustment Programs require that periphery
states reduce their public expenditures, enforcing Western control over public
services in the oppressed nations. Kenya would have to buck the trade system
that so enforces this dynamic in order to run its healthcare system
independently.

\vspace{1em}

\phantom{*}\hfill\hfill\hfill*\hfill*\hfill*\hfill\hfill\hfill\phantom{*}

\vspace{1em}

\textit{Essay truncated here for brevity.}

\iffalse

\section{Relationship to the Core}
\subsection{Historical Context}

The expansion of finance-capital in the West by its own nature necessitated
its export to new markets, and in the era of classical liberalism this meant
expansion of colonies across the world \parencite{Lenin1917}. It is of no
question that Kenya was a colony of the British Empire. The capitalist regime
utilized Kenya's resources and population for its own profit, and left when it
deemed departure to be in its best interest.

Despite Kenya flying its own flag emblazoned with the Pan-African colors, its
relationship to Western nations is primarily extractive in nature. The country
still exports far more to the West than it imports -- or even consumes on its
own -- and in many ways its policies are written by the businessmen in the
global core. Internal policies of nations are no longer determined by internal
factors, but by the whims of the global managerial class, who exist outside
the bounds of any political system and are only beholden to their financial
backers. The World Trade Organization sets binding rules on nation-states,
requiring that periphery nations expose themselves to Western monied interests
or face sanctions. The global financiers require countries to adopt structural
adjustment programs in order to receive money, or even to refinance crippling
debt. These programs often include provisions for the privatization of public
services and reduction in social benefit programs. In Kenya, privatization of
state resources and public services allowed Western financial services
corporations to reap profits by exploiting the country's impoverishment
\parencite{McMichael2004}.


\subsection{Solutions}

Within the global capitalist system, there is no way for Kenya to truly
liberate itself. The systems of real property ownership and labor relations
remain not a bug but a feature of capitalism, and Kenya must abolish its
participation in the global system that preserves its impoverishment. The
European states maintain their supremacy and standard-of-living solely through
the extraction of resources from the global periphery.

The only way this could be accomplished is through Marxist-Leninist-Maoist
class warfare. The Soviet Union was able to create a society in 1917 that
protected the rights of all people and provided healthcare, land (in the form
of state housing), and a social role to all people. As Kenya is highly
populated by peasants rather than an industrial proletariat, Mao's China
further serves as a worthwhile example of the possibilities of revolution. In
a few decades, the country managed to transition from a Western vassal state
to a world superpower. Kenya has the laborers and resources to manage on its
own, akin to the Democratic People's Republic of Korea, which has closed
itself off from the world and yet has managed to develop nuclear weapons
technology while legally still at war with the United States. With this
independent acquisition of nuclear weapons, the country has been able to bring
its oppressor, the United States, to the bargaining table. The two countries
have thus reopened tense dialogue, and true independence from the West appears
to have been successful.

However, one must look towards the former communist revolutions of Africa for
a warning. The Western powers do not look too kindly upon a country's bucking
of the capitalist system, and thwart attempts to gain independence. Without
looking at the tragedies of Latin America, Africa has been home to many
anti-capitalist movements. Libya under Colonel Muammar Gaddafi's ideological
system brought prosperity to the country and an egalitarian system unique from
the rest of the world. He was murdered in 2011 by a United States-backed coup
and the country has been in violent disarray since. Thomas Sankara promised
Burkina Faso independence from the French imperial legacy, and was deposed
quickly in a coup. Without the support of the Soviet Union, Angola's
revolutionary party was in 1992 usurped by its longtime foe, UNITA, supported
by the United States and apartheid South Africa. The protracted people's war
in Kenya would be an intense struggle, but the nation has fought for its
independence in living memory. Kenya is more than capable of challenging its
oppressors for a second time.

\fi

\printbibliography

\end{document}
